\section*{Structure of the project}
I have split the project into four files:
\begin{itemize}
  \item \textit{field.mlw} containing the provided definition of a subfield of the
    reals.
  \item \textit{interval.mlw} containing two modules \textit{BoundInterval} and
    \textit{BoundUnion} containing a full implementation of the project with respect
    to bound unions of intervals.
  \item \textit{union.mlw} containing the provided squeleton and implementing part of
    the general solution.
  \item \textit{array\_utils.mlw} containing utility functions for manipulating
    arrays.
\end{itemize}

There are many intermediary lemmas developing some ``theory'' of ordered intervals,
particularly helping to prove the unicity of the representation of each data type.
%
%% Most lemmas are simply stated and the proof load is fully handled by the provers (or
%% with a straighforward induction scheme by using \mintinline{\whyml}{let rec lemma}. In
%% this report I will not speak about such lemmas in detail (with the exception of
%% lemmas computing witness values) and will focus on the specifications of the
%% computational functions, assuming in general that if a property or invariant
%% preservation holds, it is either handled direcly by the provers or the appropriate
%% lemma was stated and proved.
%
