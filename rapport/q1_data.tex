\section{Question 1. Data structures}

For the data structures, I chose to define them in multiple levels, and lift the
operations at each step. I have thus defined:
%
\begin{enumerate}
  \item The type of (non-empty) bound intervals.
  %
  \item The type of finite unions of bound intervals.
  %
  \item The type u of possibly unbound finite union of intervals.
\end{enumerate}
%
At each step, the rationale was to define each of these objects with invariants that
guarantee a unique representation for a given mathematical set.
%

\subsection{The type of bound intervals}
Here is the type of (bound, non-empty) intervals:
\inputminted{\whyml}{why3code/interval_type.mlw}
%
It represents sets of the form
$\{ x ~|~ \texttt{lower\_bound}~\diamondsuit~x~\diamondsuit'~\texttt{upper\_bound} \}$
with
$\texttt{lower\_bound}, \texttt{upper\_bound} \in V$ and
$\diamondsuit, \diamondsuit' \in \{ \le, < \}$.
%
The invariant imposes that $\texttt{lower\_bound} \le \texttt{upper\_bound}$,
%
and that if they are equal, the interval is closed on both sides and thus represents
a singleton. These two constraints allow us to have no representation for the empty
set, which will be handled by the union layer.
%
The \mintinline{\whyml}{not_empty} is defined in \textit{field.mlw} as
\mintinline{\whyml}{of\_int 0} for convenience.

Some examples of valid intervals are $[1, 5]$, $[-3, 0[$ or $[2,2]$. Some
examples of invalid intervals are $[3, 2[$ or $[1,1[$.


\subsection{The type of unions of bound intervals}

At the pure data level, I chose to represent the union as arrays of intervals, that
is as the \mintinline{\whyml}{array interval} type.
%
The actual choice here was to have a sequential data structure, and using
arrays over algebraic lists was simply motivated by the fact that I found it more
convenient to manipulate indexes with universal quantifiers in my lemmas and predicates.
%
The actually defined data structure is enriched with invariants guaranteeing the
unique representation of a given set:
%
\inputminted{\whyml}{why3code/bound_union_type.mlw}
The \minline{(<<)} operates over intervals and is defined as follows:
%
\inputminted{\whyml}{why3code/before_interval_pred.mlw}
%
The \minline{any} is used here (and in any future use of it) to specify that this
predicate is computable.
%
It is a strict partial order, and imposing for unions of intervals to be strictly
ordered by it guarantees unicity by way of the following property: for any distinct
indices $0 \le i \neq j < \texttt{length } a$, the set $\gamma(a_i) \cup \gamma(a_j)$
is \textbf{not} an interval.

For example, $[[1, 2[,~]2, 4],~[5, 5]]$ is a valid union, and both
$[[1, 2[,~[2, 3]]$ and $[[1, 2],~[1.5, 3]$ are not valid unions because we
can fuse the two composing intervals into one, and $[[1, 1],~[0,0]$ is not valid
because it is not increasing.

\subsection{The type of general unions of intervals}
Finally, the type \minline{u} is a triple representing a union of bound intervals
with two optional numbers (of type \minline{option V.t}) encoding possible
upwards/downwards \textbf{open} unbound intervals, together with four invariants.
%
\inputminted{\whyml}{why3code/u_type.mlw}
%
The first invariant \minline{bound_unbound_disj}, combined with the fact that the
unbound parts are implicitly open, guarantees that that the optional unbound
intervals cannot be merged with an element of \minline{bu} into a single interval.
The second one \minline{unbound_unbound_disj} ensures that the two unbound intervals
are always disjoint. Here is one part of the \minline{bound_unbound_disj} predicate.
%
\inputminted{\whyml}{why3code/bound_unbound_disj_pred.mlw}
%
These two constraints allow us to have a unique representation of all sets except the
set of all elements $\mathbb{R}$, whiche can be represented by any triple of the
form $(]-\inf, x[, [[x,x]], ]x, \inf[)$ for $x\in V.t$.
%
We add the final ad-hoc invariant \minline{unique_repr_of_all} which ensures that in
that case, we have $x = 0$.

These added constraints may look unnecessarily cumbersome since it is not necessary at
all for the implementation and specification of the union and intersection
operations, but it will prove very useful for computing the inclusion and except
operators.
%
Note that all the predicates that are used in type invariants are
computational.
