\section{Question 2: the mem predicate}

I first define what it means to be part of a bound interval, which is either strictly
inside of it or equal to one of the two bounds given it is closed:

\inputminted{\whyml}{why3code/mem_interval.mlw}

I then define what it means to be part of a union of bound intervals:
\inputminted{\whyml}{why3code/mem_union.mlw}
And what it means to be part of one of the two unbound optional parts:
\inputminted{\whyml}{why3code/mem_unbound.mlw}

Finally, I define the mem predicate as being either in the bound part, or in one of
the two unbound parts:
\inputminted{\whyml}{why3code/mem.mlw}

The addition of general inclusion predicates of the structures is justified in
\autoref{q6}.
