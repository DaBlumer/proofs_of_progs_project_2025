\section{Conclusion}

\subsection{Achievements}
I have implemented all of the requested functions, and am mostly satisfied with the
end result, in particular the design choice to have a canonical representative for
every set, and to separate the work to get such a representation into conversion
functions, allowing me to have elegant subset and disjoint sound and complete
implementations.

On the other hand, I am not sure if this advantage in the simplicity of the final
implementations of the operators is worth all the (not so elegant very long in number
on lines and time to achieve) proof work that was necessary to prove that my
representation is really extensional.
%
In any case, if the completeness of the subset operator is essential, this way the
best way I could think of to achieve it with a reasonable specification.


\subsection{Experience}
Despite starting the project only after having understood all the lectures and all
the exercises, having to implement a relatively consequent specification allowed me
to get many subtleties of the why3 system (also thanks to concise yet useful
reference manual for why3); in particular I got a feel of the actual process of
writing verified code, and how to debug such code with assertions.

I also learned to trust the SMT provers as in most cases they were nice enough to
find proofs without needing intermediary hints, and in all the cases (except some
witness is necessary for the proof) where I though they were not getting something
obvious, I was actually the one to blame for some faulty algorithm or
specification.
