\section{Questions 12 to 19: increasing/decreasing operator lifting}


\subsection{The predicates (questions 12, 15, 19.1)}
Here are the predicates describing respectively stricly increasing, strictly
decreasing and increasing functions. They are written in the \minline{Intf} module.
\inputminted{\mlwhy}{why3code/op_predicates.mlw}


\subsection{Lifting operators to unions of intervals (questions 13, 16, 19.2)}

\subsubsection{For strictly increasing/decreasing}
%
The \minline{op_si} function simply applies the function to each element of the
\minline{u} data structure, in other words it applies the function to the bounds of
each interval of the union.
%
\inputminted{\whyml}{why3code/op_si.mlw}

We do not specify the completeness as
$\gamma(\text{op\_si}(\text{op}, u)) = \{\gamma(\text{op})(x)|x\in\gamma(u)\}$
because this one is only true if the function is also continuous : it requires that
for any $x < y$, op is surjective upon all of $[\text{op}(x), \text{op}(y)]$, which
is not necessarily true: the function \minline{(fun x -> if x <. 0 then x else x +. 1)}
is a counterexample.

The implementation of the op\_sd operator is similar except we exchange the
lower and upper bounds of the intervals, and we exchange the optional unbound two
intervals, and we get a sound and complete operator.

\subsubsection{For increasing}
Here we also proceed in the same manner to op\_si by applying the operator to each
interval bound, except in the final contract we can only ensure soundness.
\inputminted{\whyml}{why3code/op_nsi.mlw}
%
We see in this contract that we only guarantee that
$\forall q. q\in \gamma(u) \implies \gamma(op)(q) \in \gamma(op\_si(op,u)) $.


\subsection{Concrete operators (questions 14, 17, 18, 19)}
Once we have the lifting operator, we can implement all the desired operators in a
homogeneous way:
\inputminted{\whyml}{why3code/ops.mlw}

The operators of question 19 are also implemented similarly, for example the relu
one:
\inputminted{\whyml}{why3code/relu.mlw}
