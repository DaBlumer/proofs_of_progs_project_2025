\section{Question 8: intersection}

I implemented the intersection (on data structures without invariants)
using the union and the general fact that union and intersection commute, and so the
intersection of two unions is the union of the 9 intersections of each component of
the first with each component of the second.
%
I will detail the implementation of intersecting the two bound components, and the
other cases are simpler and implemented in an analogous manner.

\subsection{The \minline{bound_union} layer}

Again, for the intersection of two bound interval unions encoded as arrays of bound
intervals, I used the following formula based on the commutativity between union
and intersection:
%
$$
\bigcup_{i=0}^{n-1}{A_i} \cap \bigcup_{j=0}^{m-1}{B_j}
=
\bigcup_{i=0}^{n-1}{\left( A_i \cap \bigcup_{j=0}^{m-1}{B_j}\right)}
$$
%
Which gives the implementation: \inputminted{\whyml}{why3code/inter_union.mlw} The
function \minline{inter_interval} takes a union and an interval and computes their
intersection, that is it computes $A_i \cap \bigcup_{j=0}^{|B|-1}{B_j}$. This
guarantees the preservation of the loop invariant which ensures that at each step, the
result variable is equal to
%
$\bigcup_{i'=0}^{i-1}{\left( A_i \cap \bigcup_{j=0}^{m-1}{B_j}\right)}$.
%
This invariant guarantees exactly the postcondition at loop exit, and it is initially
correct as \minline{gen_mem_sub_union x bu'.a 0 0} checks if $x$ is in the empty
subarray of bu'.a and returns \minline{false}, and so does \minline{mem_union x
  result} with the initially empty \minline{result}.

The \minline{inter_interval} function is implemented with the following formula in a
similar way (modulo the conversion to ensure the \minline{bound_union} invariants).
$$
A \cap \bigcup_{j=0}^{m-1}{B_j}
=
{\bigcup_{j=0}^{m-1}{\left(A \cap B_j\right)}}
$$

\subsection{The \minline{u} layer}

First we lift the operation \minline{inter_union} to the type \minline{u}:
\inputminted{\whyml}{why3code/inter_bound_bound.mlw}

Then, we implement the other kinds of intersections in an analogous way. For example,
here is (part of) the implementation for intersecting a bound union component with a right
unbound interval:
%
\inputminted{\whyml}{why3code/inter_up_bound.mlw}
%
We see that the loop is similar to the one in \minline{inter_union} as each step adds
(applies the union to) the result of the intersection of the unbound interval with the
next element of the \minline{bound_union}, maintaining the invariant.
%
The function computing the union between a bound interval and an unbound one,
\minline{inter_up_interval}, is implemented in a straightforward way with:
%
\begin{align*}
  [a, b] \cap ]c, \infty] &=
  \begin{cases}
    [a, b]  & \text{if } c < a \\
    ]c, b]  & \text{if } a \le c < b \\
    \emptyset & \text{otherwise}
  \end{cases}
\end{align*}
%
Finally, the implementation on the intersection of two \minline{u} elements is:
\inputminted{\whyml}{why3code/inter.mlw}


