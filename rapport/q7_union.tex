\section{Question 7: Union}\label{q6}
%
The union is computed on two levels (bound unions, general unions).
The pure data-structures are used for bound and general unions as they are sufficient
and more convenient,
%
and the result is then transformed into the (unique) equivalent one that satisfies
the invariants. I expand on the specification and proof of the transformations in my
answer of questions 9-10, as those are the main motivation for working with the
structures with invariants despite the union and intersection being implemented
without them.


\subsection{The bound-intervals union level}

The implementation is made on arrays of intervals, and consists of simply appending
the second union to the first.
%
\inputminted{\whyml}{why3code/gen_union_union.mlw}
%
The contract captures exatcly the fact that $x\in \gamma(a) \iff x\in\gamma(a')$. The
replace function here is a generic array operation I defined in the \minline{ArrayUtils}
module, and which specifies that the result is the first array where a part of it
(delimited by the third and fourth arguments) is replaced by the second array.
%
It is equipped with the following lemma, which guarantees that
$\gamma($\minline{replace a a' i }$) = (\gamma(a) \cup$
$\gamma(a'))\setminus\gamma(a[i,...,j-1])$:
\inputminted{\whyml}{why3code/replace_lemma.mlw}

To obtain a result of type \minline{bound_interval} from an array, we apply two
conversion functions:
\inputminted{\whyml}{why3code/union_union.mlw}

Here is the signature of the two conversion functions, capturing that we preserve the
same notion of concrezitation to a set throughout the conversions:
\inputminted{\whyml}{why3code/bound_convs_contracts.mlw}

The implementation and proof of these functions, and the definition of the
\minline{lax_bound_union} intermediate type is described as the computational part of
questions 9-10.

\subsection{The general union level}

Again, we implement the union on a version without invariant, with the following
straightforward implementation that takes the union of each part separately.

  \begin{algorithmic}
     \Require { (lo, bu, up), (lo', bu', up') : (option V.t, bound\_union, option V.t) }
     \Ensure { $\gamma($res\_lo, res\_bu, res\_up$) =\gamma($lo,bu,up
               $)\cup\gamma($lo',bu',up'$)$ }
     \State res\_bu $\gets$ union\_union bu bu'
     \State res\_lo $\gets$ max lo lo'  (w.r.t V.(<) extended with None $<$ Some \_)
     \State res\_up $\gets$ min lo lo' (w.r.t V.(<) extended with Some \_ $<$ None)
  \end{algorithmic}

This is sound and complete because each triple represents a finite union of its three
components and the union is commutative.

The final contract looks like this:
\inputminted{\whyml}{why3code/union_contract.mlw}
It implements the pseudo-code above, then converts the triple into one satisfying the
type invariants of u using the function fix\_untyped which we
detail in questions 9-10 and which has the following contract:
\inputminted{\whyml}{why3code/fix_untyped_contract.mlw}
