\section{Questions 9 to 11: Subset and intersection}

I start in \autoref{q9-11:impl} by showing the actual implementation of the functions
which is elementary and relies on the fact that $A\cap B=A \iff A \subset B$.
%
For this to work we need our representation to be extensional: that is we need for
each representable set to have only one encoding within our types, and it is exactly
the motivation behind our type invariants. I discuss the proof of this fact in
\autoref{q9-11:ext}
%
Finally, since the real computation that matters here is the conversion functions
that compute the equivalent result satisfying the invariants, I discuss these in
\autoref{q9-11:conv}.


\subsection{Implementation and soundness (questions 9 and 10)}\label{q9-11:impl}

First, some computable predicates that say that the data structures are equal pointwise.
\inputminted{\whyml}{why3code/weak_eq.mlw}
And the two implementations:
\inputminted{\whyml}{why3code/incl_disj.mlw}

The soundness is trivial here as were are litteraly requestion for the two sets to be
represented in the same way.

\subsection{Completeness: proof of the unicity of representation (question
  11)}\label{q9-11:ext}

To show the completeness of these two functions, I showed the following two lemmas,
stating that if $\gamma(u)=\gamma(v)$ then $u = v$.
%
\inputminted{\whyml}{why3code/extentionality_lemmas.mlw}
%
To prove these lemmas, I had to show many auxiliary ones that all essentially provide
witnesses $x \in u \setminus v$ anytime we have that $u = v$, .
%
The whole process quite long and not very elegant, and is approximatively done in 400
lines. As an example, the following lemma provides an arbitrarily big element of an
unbound union.
%
\inputminted{\whyml}{why3code/ghost_example.mlw}

\subsection{Conversion functions}\label{q9-11:conv}
We are able to make simple computations without preserving type invariants for the
union and intersection operations thanks because we separate those operations from
the burden of keeping the invariants. This is possible because the \minline{mem} we
dedined does not depend on invariants and remains valid.

\subsubsection{Converting an array to a \minline{bound_union}}

I do this in two steps: first I convert the array to an intermediate type
\minline{lax_bound_union} which ensures that the intervals are sorted by their lower
bounds, then I transform this intermediate structure with the following function:
%
\inputminted{\whyml}{why3code/lax_to_strict.mlw}
%

This function operates on an array of intervals sorted by their lower bounds, and at
each loop step with index $0\le i < |lbu.a| - 1$, it can do one of two things:
\begin{itemize}
  \item if the interval at $a_i$ occures strictly before $a_{i+1}$ (for example
    $[0,1] << [1.5,2]$), it advances by incrementing $i$, and by preserving the
    third invariant of the loop and making the variant decrease.
  \item otherwise, this means the two intervals can be merged (for example we may
    have $([0,1],]1,2])$ or $([0,2],[1,3])$. In that case we merge the intervals into
    one that has the same elements, thus preserving the fourth invariant and making
    the variant decrease as the array size descreases by 1.
    This is ensured by the contract of the \minline{lax_merge} function.
\end{itemize}
Finally, first invariant is true and maintained as $i$ starts by being positive, and
the second one (where \minline{<==} is the order by lower bound) is true at first
because of the type invariant of tmp, and is preserved because every change of
elements preserves the sorting by lower bound. The third and fourth invariants are
trivially true at start, and the postcondition is equivalent to the fourth invariant
at loop exit.


\subsubsection{Converting a triple in \minline{(option V.t, bound_union, option V.t)}
  to \minline{u}}
The process is analogous to that of the preceding conversion, and without going into
details, here is the contract of the final function:
%
\inputminted{\whyml}{why3code/fix_untyped.mlw}
